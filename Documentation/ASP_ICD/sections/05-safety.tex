% 05-safety.tex - Section 5: Safety Interface

\chapter{Safety Interface}
\label{sec:safety}

\section{Overview}

The ASP subsystem includes several safety features to protect equipment from damage due to temperature or power supply anomalies.

\section{Temperature Monitoring}

The ASP continuously monitors temperature via sensors associated with the power supply units.

\subsection{Temperature Thresholds}

The temperature thresholds and associated actions are listed in Table~\ref{tab:temp-thresholds}.

\begin{table}[htbp]
    \centering
    \caption{Temperature Thresholds}
    \label{tab:temp-thresholds}
    \begin{tabular}{lll}
        \toprule
        \textbf{Threshold} & \textbf{Value} & \textbf{Action} \\
        \midrule
        Minimum & \SI{0.0}{\degreeCelsius} & Under-temperature warning/error \\
        Warning & \SI{40.0}{\degreeCelsius} & System enters WARNING state \\
        Maximum & \SI{45.0}{\degreeCelsius} & System enters ERROR state; power supplies turned off \\
        \bottomrule
    \end{tabular}
\end{table}

\subsection{Temperature Response}

\begin{enumerate}
    \item \textbf{Warning Condition}: When any temperature sensor exceeds the warning threshold (\SI{40}{\degreeCelsius}), the system enters WARNING state. This is a cautionary indicator.

    \item \textbf{Over-Temperature}: If temperatures exceed the maximum threshold (\SI{45}{\degreeCelsius}) for three consecutive monitoring cycles, the system:
    \begin{itemize}
        \item Enters ERROR state
        \item Turns off ARX power supply
        \item Turns off FEE power supply
        \item Sets the ready flag to false
    \end{itemize}

    \item \textbf{Under-Temperature}: If temperatures drop below the minimum threshold (\SI{0}{\degreeCelsius}) for three consecutive monitoring cycles, the system enters ERROR state but does not turn off power supplies.
\end{enumerate}

\section{Power Supply Monitoring}

The ASP monitors power supply status for fault conditions.

\subsection{Monitored Conditions}

The power supply fault conditions and their responses are listed in Table~\ref{tab:psu-faults}.

\begin{table}[htbp]
    \centering
    \caption{Power Supply Fault Conditions}
    \label{tab:psu-faults}
    \begin{tabular}{lp{8cm}}
        \toprule
        \textbf{Condition} & \textbf{Response} \\
        \midrule
        Over Temperature & Shut down affected power supply; enter ERROR state \\
        Over Current & Shut down affected power supply; enter ERROR state \\
        Over Voltage & Shut down affected power supply; enter ERROR state \\
        Under Voltage & Shut down affected power supply; enter ERROR state \\
        Module Fault & Shut down affected power supply; enter ERROR state \\
        \bottomrule
    \end{tabular}
\end{table}

\section{Control Board Monitoring}

The system periodically verifies that the ARX Control Board USB device is present and responding. If the control board disappears:
\begin{itemize}
    \item System enters ERROR state
    \item INFO field reports device not found
    \item Ready flag is set to false
\end{itemize}

\section{Board Configuration Monitoring}

The chassis status monitoring thread periodically verifies that ARX boards maintain their SPI port configuration. The thread reads a known configuration register from the RS485 bus and checks that its value matches the expected configuration pattern. If the bus has lost its SPI port setup (e.g., from a USB reset or power glitch), the register value will differ, indicating that communication with the boards is no longer possible. If a board loses configuration:
\begin{itemize}
    \item System enters ERROR state
    \item INFO field reports which antennas are affected
    \item Ready flag is set to false
\end{itemize}

\begin{calloutBox}{Rev H System Difference}{orange}
On the legacy Rev H system, board configuration monitoring uses a timestamp-based mechanism rather than SPI register checks. During initialization, a reference timestamp is written to all boards via the RS-485 \cmd{STIM} command. The monitoring thread periodically queries each board for its stored time using the \cmd{GTIM} command. If any board has reset or lost power, its stored time will differ from the reference. This approach can identify exactly which individual board(s) have failed, rather than detecting a whole RS485 bus failure.
\end{calloutBox}

\section{Error Recovery}

To recover from most error conditions, the system must be re-initialized using the \cmd{INI} command. The operator should:
\begin{enumerate}
    \item Address the underlying cause of the error (temperature, power supply fault, etc.)
    \item Issue a \cmd{SHT} command if the system is not already shut down
    \item Issue an \cmd{INI} command with the appropriate board count
\end{enumerate}
